%% LyX 2.2.1 created this file.  For more info, see http://www.lyx.org/.
%% Do not edit unless you really know what you are doing.
\documentclass[oneside,english]{amsart}
\usepackage[T1]{fontenc}
\usepackage[latin9]{inputenc}
\usepackage{geometry}
\geometry{verbose,tmargin=1in,bmargin=1in,lmargin=1in,rmargin=1in}
\usepackage{amstext}
\usepackage{amsthm}
\usepackage{amssymb}

\makeatletter
%%%%%%%%%%%%%%%%%%%%%%%%%%%%%% Textclass specific LaTeX commands.
\numberwithin{equation}{section}
\numberwithin{figure}{section}
\theoremstyle{plain}
\newtheorem{thm}{\protect\theoremname}
  \theoremstyle{definition}
  \newtheorem{defn}[thm]{\protect\definitionname}
  \theoremstyle{definition}
  \newtheorem{xca}[thm]{\protect\exercisename}
  \theoremstyle{plain}
  \newtheorem{question}[thm]{\protect\questionname}

\makeatother

\usepackage{babel}
  \providecommand{\definitionname}{Definition}
  \providecommand{\exercisename}{Exercise}
  \providecommand{\questionname}{Question}
\providecommand{\theoremname}{Theorem}

\begin{document}

\title{MATH 321 Day 14 - Power Sets and Cantor's Theorem}
\maketitle
\begin{defn}
Given a set $A$, the \textbf{power set }$P(A)$ is the collection
of all subsets of $A$. $P(A)$ is a set of subsets of $A$.
\end{defn}
\begin{itemize}
\item We care about power sets because, given a set $A$, $P(A)$ is ``much
bigger'' than $A$ (and thus 
\end{itemize}
\begin{xca}
~
\begin{enumerate}
\item Let $A=\{a,b,c\}$. List the eight elements of $P(A)$. (Do not forget
that $\emptyset$ is considered to be a subset of every set.)
\begin{itemize}
\item $P(A)=\{\emptyset,\{a\},\{b\},\{c\},\{a,b\},\{a,c\},\{b,c\},A\}$
\end{itemize}
\item If $A$ is finite with $n$ elements, show that $P(A)$ has $2^{n}$
elements.
\begin{itemize}
\item Let $A=\{x_{1},\dots,x_{n}\}$ and $S\subseteq A$. Then, for all
$a\in A$, either $a\in S$ or $a\notin S$. This means that we can
represent $S$ by a string of numbers 
\[
a_{1},a_{2},\dots,a_{n}
\]
where for all $i$, 
\[
a_{i}=\begin{cases}
0 & \text{if }x_{i}\notin S\\
1 & \text{if }x_{i}\in S.
\end{cases}
\]
Each $\{a_{i}\}_{i=1}^{n}$ represents a unique subset of $A$, since
a subset is uniquely determined by which elements are in it. The number
of strings $\{a_{i}\}_{i=1}^{n}$ is $2^{n}$. Hence $P(A)$ has $2^{n}$
elements.
\end{itemize}
\end{enumerate}
\end{xca}
~
\begin{xca}
~
\begin{enumerate}
\item Using the particular set $A=\{a,b,c\}$, exhibit two different 1-1
mappings from $A$ into $P(A)$.
\begin{enumerate}
\item $a\mapsto\{a\}$ for all $a\in A$
\item $a\mapsto A\setminus\{a\}$
\end{enumerate}
\item Letting $C=\{1,2,3,4\}$, produce an example of a 1-1 map $g:C\to P(C)$.
\begin{enumerate}
\item $n\mapsto\{n\}$ for all $n\in C$.
\end{enumerate}
\item Explain why, in parts (a) and (b), it is impossible to construct mappings
that are \emph{onto}.
\begin{enumerate}
\item Since $A,C,P(A),P(C)$ are finite sets and $|A|<|P(A)|$, $|C|<|P(C)|$,
once we choose where each of the elements of $A$ or $C$ go, there
still remain $2^{3}-3$ elements of $P(A)$ and $2^{4}-4$ elements
of $P(C)$ which aren't mapped to.
\end{enumerate}
\end{enumerate}
\end{xca}
\begin{itemize}
\item Cantor's Theorem says this is impossible even for infinite sets:
\end{itemize}
\begin{thm}
Given any set $A$, there does not exist a function $f:A\to P(A)$
that is onto.
\end{thm}
\begin{proof}
Assume for contradiction that $f:A\to P(A)$ is onto. For each $a\in A$,
$f(a)\subseteq A$. The assumption that $f$ is onto means that every
subset of $A$ appears as $f(a)$ for some $a\in A$. To arrive at
a contradiction, we will produce a subset $B\subseteq A$ that is
not equal to $f(a)$ for any $a\in A$. 

For each element $a\in A$, consider the subset $f(a)$. If $a\notin f(a)$,
we include $a$ in our set $B$. More precisely, let
\[
B=\{a\in A:a\notin f(a)\}.
\]
\begin{xca}
Return to the particular functions constructed in the previous exercise
and construct the subset $B$ that results using the previous rule.
In each case, note that $B$ is not in the range of the function used.
\end{xca}
Because we have assumed that $f:A\to P(A)$ is onto, it must be that
$B=f(a^{\prime})$ for some $a^{\prime}\in A$. The contradiction
arises when we consider whether $a^{\prime}\in B$.
\begin{xca}
~
\begin{enumerate}
\item First, show that the case $a^{\prime}\in B$ leads to a contradiction.
\item Now, finish the argument by showing that the case $a^{\prime}\notin B$
is equally unacceptable.
\end{enumerate}
\end{xca}
\end{proof}
\begin{itemize}
\item Cantor's Theorem implies that there's no function from $\mathbb{N}$
to $P(\mathbb{N})$; in other words, $P(\mathbb{N})$ is uncountable!
\end{itemize}
\begin{question}
How does the cardinality of the uncountable set $P(\mathbb{N})$ compare
to that of the uncountable set $\mathbb{R}$?
\end{question}
\begin{itemize}
\item In fact, one can show that $P(\mathbb{N})\sim S\sim(0,1)\sim\mathbb{R}$,
where $S$ is the set of sequences of $0$s and $1$s. Hence, $P(\mathbb{N})\sim\mathbb{R}$.
\end{itemize}
\begin{xca}
{[}take-home challenge!{]} Prove that $S\sim(0,1)$ by constructing
1-1 functions $f:S\to(0,1)$ and $g:(0,1)\to S$. It's a fact that,
if we can construct such functions, then the two sets they map between
are in 1-1 correspondence.
\end{xca}
~
\begin{xca}
Answer each of the following by establishing a 1-1 correspondence
with a set of known cardinality.
\begin{enumerate}
\item Is the set of all functions from $\{0,1\}$ to $\mathbb{N}$ countable
or uncountable?
\item Is the set of all functions from $\mathbb{N}$ to $\{0,1\}$ countable
or uncountable?
\item Given a set $B$, a subset $\mathcal{A}$ of $P(B)$ is called an
\emph{antichain }if no element of $\mathcal{A}$ is a subset of any
other element of $\mathcal{A}$. Does $P(\mathbb{N})$ contain an
uncountable antichain?
\end{enumerate}
\end{xca}

\end{document}
