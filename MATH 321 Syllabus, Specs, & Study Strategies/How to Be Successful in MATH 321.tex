\documentclass[12pt]{article}
\usepackage{geometry}                % See geometry.pdf to learn the layout options. There are lots.
\geometry{letterpaper}                   % ... or a4paper or a5paper or ... 
%\geometry{landscape}                % Activate for for rotated page geometry
%\usepackage[parfill]{parskip}    % Activate to begin paragraphs with an empty line rather than an indent
\usepackage{graphicx}
\usepackage{helvet}
\renewcommand{\familydefault}{\sfdefault}
\usepackage{amssymb}
\usepackage{epstopdf}
\usepackage{hyperref}
\usepackage{lmodern}
\DeclareGraphicsRule{.tif}{png}{.png}{`convert #1 `dirname #1`/`basename #1 .tif`.png}
\title{How to Be Successful in MATH 321}
\author{Kenan Ince}
\date{Fall 2017}                         % Activate to display a given date or no date

\begin{document}
\maketitle

(\textit{Adapted from \href{https://docs.google.com/document/d/1NeRgy4sWs7ZPfCK7kTCBqNOhEy-0lMbRIO0A5jwdyNI/edit}{Robert Talbert} and \href{https://www.amazon.com/Think-About-Analysis-Lara-Alcock/dp/0198723539}{Lara Alcock}})

I want you to be successful in the course and will be working hard to make sure you always have a clear path to success. But I cannot walk that path for you! It will take considerable effort on your part. To do this and be successful, you will need to do the following: 

\begin{enumerate} 
\item \textbf{Prepare diligently outside of class and come to class ready to work.}
	\begin{enumerate}
	\item Get an early start on all class preparation assignments. These are usually posted several days in advance. 
	\item DON'T PROCRASTINATE on class preparation by waiting until the night before they are due. The ideas in MATH 321 are abstract and take time to settle. You can't force understanding of them on a tight time frame. 
	\item If you get stuck preparing for class, ask for help -- email me, work with a friend, whatever. 
	\item Review your preparatory work and prepare questions before arriving at class. 
	\end{enumerate}
\item \textbf{Actually come to class.}
\item \textbf{When in class, be engaged and active in your learning.} This means: 
	\begin{enumerate}
	\item Use what you learned in your class preparation to extend yourself to a harder problem. 
	\item Make effective use of the class environment to ask questions of your friends and me, and seek help where you need it and give help to others. 
	\item Choose to avoid inappropriate distractions in class such as Facebook or texting. 
	\end{enumerate}
\item \textbf{Also be engaged and active in your learning after class.} This means: 
	\begin{enumerate}
	\item After every class session, spend between $30$ and $60$ minutes close-reading and thinking critically about your class notes. Aim for good self-explanations (see \href{https://www.dropbox.com/s/ttuwhub8wriuymw/Alcock Ch. 3 - Proofs.pdf?dl=0}{Ch. 3 of Alcock's \textit{How to Think About Analysis}} for details on self-explanations) of each definition, theorem, and proof.
	\item After close-reading, if there's anything you don't understand after spending a few minutes thinking about it, write it on a piece of paper entitled "Questions about Advanced Calc", and make a note of where this thing is and what exactly you don't understand. Be precise--sometimes nailing down the problem allows you to sort it out and, if it doesn't, you will have a specific note to come back to so you don't lose the thinking you've already done. Plus, you can bring your list to class or to office hours and ask me; I'm glad to help!
	\item Come to office hours, make appointments, or send email when you get stuck on a problem or have a question about something. I expect you to come to office hours; it's not something you do as a last resort. 
	\item Spend at least 2 hours outside of class on MATH 321 work for every hour you spend in class, so roughly 8 hours a week just on MATH 321. 
	\item The time you spend on MATH 321 is to spent purposefully, with a plan for what you will do and when you will do it as well as a plan for getting help if you get stuck. 
	\end{enumerate}
\item \textbf{Adopt a "growth mindset" for your intellectual development.} According to Carol Dweck, the psychologist who coined this term, those with a "fixed mindset" believe their basic qualities such as intelligence or mathematical skill are fixed quantities. On the other hand those with a "growth mindset" believe that these basic qualities can be improved through dedication and hard work, and when they fail at something, they take it as a learning opportunity and get better by learning from their mistakes. (More at \url{http://edglossary.org/growth-mindset/}.) 
\item \textbf{Avoid wasting time while doing homework or studying.}
	\begin{enumerate}
	\item Start homework assignments as early as possible, preferably the day you receive them. Each homework assignment will be due one week from the day it's assigned, so if you start early you'll have the chance to really mull over the problems. Mathematical problem-solving and proof takes time and sometimes several spaced-out attempts.
	\item When working on homework, begin with a "first pass" in which you spend about ten minutes on each problem. Some problems you will be able to finish in this time, especially if they involve routine warm-up exercises or direct applications of an idea that you've just studied. (In such cases, see how much you can do without looking at your notes--this might take slightly longer but, if you can construct or reconstruct something for yourself, you will remember it better in the long run.)
	\item Other problems you will not be able to finish in ten minutes. If you are making good progress, you might want to carry on for a bit longer. If, on the other hand, you're stuck, and if you've tried a few sensible things to get unstuck, make a note on your "Questions about Advanced Calc" sheet and move on--those other problems are still waiting.
	\item Now, I said "at a first pass" because I think problem solving in Advanced Calc should be a multiple-pass task. You want to give it a try, then have a break for a few hours or a day, then give it another try. Magical things will sometimes happen in the break--your brain will make new connections and you'll see new ways forward. So you probably want to break up your study into at least a couple of blocks. Indeed, you should do that anyway, because thoughtful study is intellectually effortful--if you decide to spend four hours at a stretch studying Advanced Calc, you may waste the last two simply because you will run out of energy.
	\item Keep an eye on your "Questions about Advanced Calc" list. Sometimes, working on problems will make you think about an idea in a different way, and you'll be able to cross off something that you added when studying your notes. Sometimes, when you've had a break for a couple of days, a quick re-read of your notes will make something click, and you'll be able to finish a problem and cross that off too. After that, here's what I'd do. 
		\begin{enumerate}
		\item First, get together with a friend or two and work systematically through your respective lists. Everyone thinks a bit differently, so you will probably be able to fill some gaps for one another. Doing this will also force you to speak about Advanced Calc, helping you to become fluent in talking about the concepts and explaining your arguments. Fluency is important, so don't worry if you trip over your words at first. Just have another try--you will only get more confident with practice. Sharing ideas will also help you to become a good mathematical listener. Pay close attention to what your friends are saying and, if you are not sure you understand, say so, and try to specify what is confusing you. Doing this will help your friends to articulate their thoughts more clearly. Again, this is a valuable skill that will help all of you to speak more confidently to professors and other tutors. 		
		\item Of course, as in individual work, don't get obsessed--if you can't figure something out between you in a reasonable amount of time, perhaps your effort would be better spent elsewhere. Once you've shared your knowledge with friends, take your remaining questions to me (you can always go to me first, of course, but consider the above issues about developing communication skills).
		\item When you come see me, take your list and your problem sheets and all your relevant notes, and make sure that your list has page or section or question numbers on it--you want to be able to find everything with minimal fuss. If seeing me involves arranging a specific meeting, consider bringing your classmates with whom you've been working--that should make the process more efficient. And do not be shy about asking questions, even if you have a long list. Trust me, a student asking specific questions from a well-organized list is always impressive.
		\end{enumerate}
	\end{enumerate}
\end{enumerate}
 
Overall, I don't really expect anyone to behave precisely this way. It's important that you reflect occasionally on how thing are going, and be ready to adjust. If you need longer to study your notes, adjust your timings; if you need to study for a test in another subject, cut back the essentials in Advanced Calc for a week. And of course, if you're really into a problem, stare into space and think about it for hours, if you like. The advice here should be thought of as a useful place to start, and as a way to develop a routine that will keep you going through the challenging weeks.

I want you to be successful! I believe that you can all succeed in MATH 321 by cooperating with me and thinking carefully about the material and your study habits.

\end{document}